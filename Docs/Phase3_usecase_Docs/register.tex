\documentclass[a4paper]{article}
\usepackage{hyperref}
\usepackage{amsmath}
\usepackage{tikz}
\usetikzlibrary{shapes, arrows}
\usepackage{xepersian}
\usepackage{graphicx}
\usepackage{ulem}
\setlatintextfont[Scale=0.8]{Times New Roman}
\settextfont{XB Niloofar}
\setdigitfont{XB Niloofar}

\title{\textbf{ثبت‌ نام}}
\author{نسخه‌ی ۱}
\date{}
\begin{document}

\begin{center}
به نام خدا
\end{center}
{\let\newpage\relax\maketitle}
\maketitle
\rule{\textwidth}{2pt}
\section{بازیگران}
کاربر عام: هر کاربری که به صفحه‌ی ثبت‌ نام دسترسی داشته باشد. 
\section{شرح}
این مورد کاربردی معرف رویدادی است که در طی آن یک کاربر عام می‌تواند در سامانه ثبت نام کرده و از امکانات سامانه استفاده کند.

برای این کار کاربر باید اطلاعات زیر را در سامانه به درستی وارد کند:
\begin{itemize}
\item
نام
\item
نام خانوادگی
\item
رایانامه
\item
شماره ملی
\item
شناسه‌ی کاربری
\item
رمز عبور
\item
تکرار رمز عبور
\item
شماره‌ی تلفن همراه
\item
\lr{captcha}
\end{itemize}
زمانی که کاربر اطلاعات فوق را به درستی وارد کرد. با زدن دکمه‌ی ثبت نام در صورت قابل پذیرش بودن اطلاعات حساب کاربری برایش ایجاد می‌گردد.
\section{سناریو}
\subsection{اعمال بازیگر}
\begin{enumerate}
\item
مراجعه‌ی کاربر به صفحه‌ی ثبت نام سامانه.
\item
پر کردن فرم ثبت نام و فشردن دکمه‌ی ثبت نام.
\end{enumerate}
\subsection{پاسخ سیستم}
پس از دو مرحله‌ی بالا بسته به نحوه‌ی پرکردن فرم توسط کاربر سیستم پاسخ‌های مختلفی را از خود نشان می‌دهد:
\begin{itemize}
\item
در صورت خالی بودن هر یک از فیلدها سیستم پیامی مربوط به پرکردن آن بخش به کاربر می‌دهد. و از کاربر می‌خواهد آن بخش را کامل کند.
\item
در صورت مناسب نبودن فرمت رایانامه سیستم به کاربر اطلاع می‌دهد که رایانامه‌اش را به صورت صحیح وارد نکرده و باید آن را اصلاح کند.
\item
در صورت یکسان نبودن رمز عبور با تکرار آن سیستم از کاربر می‌خواهد آن‌ها را دوباره وارد کند.
\item
در صورت تکراری بودن نام‌کاربری سیستم از کاربر می‌خواهد که نام دیگری را برای خود انتخاب کند.
\item
شماره‌ی ملی باید از ۱۰ رقم تشکیل شده باشد در غیر این صورت پیامی متناسب به کاربر نمایش داده می‌شود.
\item
در صورتی که در فیلد‌ها دستور SQL وجود دارد باید خطا در نحوه‌ی پر کردن فرم به اطلاع کاربر برسد.
\item
در صورت عدم تطابق \lr{captcha} با پاسخ کاربر سیستم از کاربر می‌خواهد که آن را دوباره وارد کند.
\item
در نهایت در صورتی که تمام موارد بالا رعایت شده بود و فرم اطلاعات کاربر صحت سنجی شد. سیستم کاربر را ثبت نام می‌کند.
\end{itemize}
\end{document} 
