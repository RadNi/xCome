\documentclass[a4paper]{article}
\usepackage{hyperref}
\usepackage{amsmath}
\usepackage{tikz}
\usetikzlibrary{shapes, arrows}
\usepackage{xepersian}
\usepackage{graphicx}
\usepackage{ulem}
\setlatintextfont[Scale=0.8]{Times New Roman}
\settextfont{XB Niloofar}
\setdigitfont{XB Niloofar}

\title{\textbf{ارتباط با مدیریت}}
\author{نسخه‌ی ۱}
\date{}
\begin{document}

\begin{center}
به نام خدا
\end{center}
{\let\newpage\relax\maketitle}
\maketitle
\rule{\textwidth}{2pt}
\section{بازیگران}
کاربر عام: هر کاربری که به صفحه‌ی ثبت‌ نام دسترسی داشته باشد. 
\section{شرح}
این مورد کاربردی معرف رویدادی است که در طی آن یک کاربر عام می‌تواند با استفاده از سامانه با مدیریت مستقیما ارتباط گرفته و به او پیام دهد.

برای این کار کاربر باید اطلاعات زیر را در سامانه به درستی وارد کند:
\begin{itemize}
\item
نام
\item
نام خانوادگی
\item
رایانامه
\item
متن پیام
\item
\lr{captcha}
\end{itemize}
زمانی که کاربر اطلاعات فوق را به درستی وارد کرد. با زدن دکمه‌ی ارسال پیام، پیام خود را به مدیریت ارسال می‌کند.
\section{سناریو}
\subsection{اعمال بازیگر}
\begin{enumerate}
\item
مراجعه‌ی کاربر به صفحه‌ی ارتباط با ما در سامانه.
\item
پر کردن فرم ارتباط با مدیریت و ارسال.
\end{enumerate}
\subsection{پاسخ سیستم}
پس از دو مرحله‌ی بالا بسته به نحوه‌ی پرکردن فرم توسط کاربر سیستم پاسخ‌های مختلفی را از خود نشان می‌دهد:
\begin{itemize}
\item
در صورت خالی بودن فیلدهای پیام و \lr{captcha} به کاربر پیام مناسب جهت پر کردن این فیلدها به کاربر نمایش می‌دهد.
\item
در صورت مناسب نبودن فرمت رایانامه سیستم به کاربر اطلاع می‌دهد که رایانامه‌اش را به صورت صحیح وارد نکرده و باید آن را اصلاح کند.
\item
در صورتی که در فیلد‌ها دستور SQL وجود دارد باید خطا در نحوه‌ی پر کردن فرم به اطلاع کاربر برسد.
\item
در صورت عدم تطابق \lr{captcha} با پاسخ کاربر سیستم از کاربر می‌خواهد که آن را دوباره وارد کند.
\item
در نهایت در صورتی که تمام موارد بالا رعایت شده بود و فرم اطلاعات کاربر صحت سنجی شد. سیستم پیام کاربر را برای مدیریت ارسال می‌کند.
\end{itemize}
\end{document} 
